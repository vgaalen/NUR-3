\section{Satellite galaxies around a massive central – part 2}
Code of shared modules:
\lstinputlisting[firstline=1, lastline=305]{ex1.py}

\newpage
\subsection*{a}
Here the goal is to find the x-value for which the number density of satellite galaxies is the highest (using the same formula as in hand-in 2).
I chose to use a golden ratio search algorithm here, as it can find a minimum (or maximum) within a bracket without any option to jump out.
As the exercise gave a specific range to search in, this algorithm is well suited here.

This gave the following solution:
\lstinputlisting{output/ex1a.txt}

The code used for this subquestion is:
\lstinputlisting[firstline=309, lastline=324]{ex1.py}

\subsection*{b}
For this exercise a dataset containing the positions of millions of satellite galaxies relative to their cluster center is used to fit a number density profile.
The profile from the second hand-in is used again, but now the a, b, and c parameters are fitted to the data.
This is done for 5 different bins for halo mass.

In order to compare the discrete galaxy positions with number densities, these galaxies are binned based on their distance to the cluster center.
I decided to use 20 bins scaled logarithmically between x=0.1 and x=2.5. 
Figure \ref{fig:ex1b_0} shows that the satellite galaxy distribution falls off very quickly towards x=2.5. 
Taking a larger $x_{max}$ would therefore only add empty bins increasing computation time.
The lower bound was taken such that about half of the bins would fall into the model's power-law regime and half in the exponential regime.
The number of bins was chosen as to accurately describe the drop-off with as little bins as possible.
As a lower number of bins gives us more galaxies in each bin, the error margin on these values should decrease.
20 bins gave a good tradeoff between these arguments.

A downhill simplex minimization algorithm is then used to find the best fit for the parameters based on a $\chi^2$.
There are no errors included in the dataset, therefore $\sigma^2 = N$ was used here (which is the variance of a Poisson distribution).

For the Downhill Simplex's initial points I took a random sample (using the RNG from hand in 2) of points within a reasonable range (4 points to be precise as the Downhill Simplex operates in 3D).
This range was determined by eye using figure \ref{fig:Ndx} balancing a good view of the parameter space and giving a good initial guess for our dataset.
This gave $1\leq a_{init}\leq 10$, $0.15\leq b_{init}\leq 0.75$, and $1\leq c_{init}\leq 5$.
These points are checked for degeneracy before they are input to the minimization algorithm.
This is done by shifting the axes to one of the points and testing whether there are two points with the same position angle(s) (multiple angles in higher dimensions).
If there are no duplicates found when shifting the axes to all the points in the simplex, the set is non-degenerate.

\begin{figure}
\centering
\begin{tabular}{ccc}
    \includegraphics[width=0.3\linewidth]{plots/Ndx_a.pdf} & \includegraphics[width=0.3\linewidth]{plots/Ndx_b.pdf}&\includegraphics[width=0.3\linewidth]{plots/Ndx_c.pdf} \\
    (a) & (b) & (c)
\end{tabular}
\caption{Satellite galaxy distribution for different values for the parameters a, b, and c (in their respective subfigures).}
\label{fig:Ndx}
\end{figure}

The best fit values found are:
\lstinputlisting{output/ex1b.txt}

These give the profiles that can be seen in figures \ref{fig:ex1b_0}-\ref{fig:ex1b_4}.
\begin{figure}
    \centering
    \includegraphics[width=0.8\textwidth]{plots/1b_0.png}
    \caption{Number density profile and fit for haloes of approximate mass $10^{11}$ Msun/h. The fit was constructed using Downhill Simplex minimizing a $\chi^2$}
    \label{fig:ex1b_0}
\end{figure}
\begin{figure}
    \centering
    \includegraphics[width=0.8\textwidth]{plots/1b_1.png}
    %\caption{Number density profile and fit for haloes of approximate mass $10^{12}$ Msun/h.}
    \caption{The same as \ref{fig:ex1b_0}, but for halos of mass $\sim 10^{12}$ Msun/h.}
    \label{fig:ex1b_1}
\end{figure}
\begin{figure}
    \centering
    \includegraphics[width=0.8\textwidth]{plots/1b_2.png}
    %\caption{Number density profile and fit for haloes of approximate mass $10^{13}$ Msun/h.}
    \caption{The same as \ref{fig:ex1b_0}, but for halos of mass $\sim 10^{13}$ Msun/h.}
    \label{fig:ex1b_2}
\end{figure}
\begin{figure}
    \centering
    \includegraphics[width=0.8\textwidth]{plots/1b_3.png}
    %\caption{Number density profile and fit for haloes of approximate mass $10^{14}$ Msun/h.}
    \caption{The same as \ref{fig:ex1b_0}, but for halos of mass $\sim 10^{14}$ Msun/h.}
    \label{fig:ex1b_3}
\end{figure}
\begin{figure}
    \centering
    \includegraphics[width=0.8\textwidth]{plots/1b_4.png}
    %\caption{Number density profile and fit for haloes of approximate mass $10^{15}$ Msun/h.}
    \caption{The same as \ref{fig:ex1b_0}, but for halos of mass $\sim 10^{15}$ Msun/h.}
    \label{fig:ex1b_4}
\end{figure}

The code used for this subquestion is:
\lstinputlisting[firstline=326, lastline=417]{ex1.py}

\subsection*{c}
This exercise is very similar to 1b, except that the assumption of a Gaussian distribution around the density profile is dropped.
As it is a distribution around a number density, a Poisson distribution is more accurate.

The best fit values found are:
\lstinputlisting{output/ex1c.txt}

These give the profiles that can be seen in figures \ref{fig:ex1c_0}-\ref{fig:ex1c_4}.
\begin{figure}
    \centering
    \includegraphics[width=0.8\textwidth]{plots/1c_0.png}
    \caption{Number density profile and fit for haloes of approximate mass $10^{11}$ Msun/h. The fit was constructed using Downhill Simplex minimizing a Poissonian log likelihood}
    \label{fig:ex1c_0}
\end{figure}
\begin{figure}
    \centering
    \includegraphics[width=0.8\textwidth]{plots/1c_1.png}
    %\caption{Number density profile and fit for haloes of approximate mass $10^{12}$ Msun/h.}
    \caption{The same as \ref{fig:ex1c_0}, but for halos of mass $\sim 10^{12}$ Msun/h.}
    \label{fig:ex1c_1}
\end{figure}
\begin{figure}
    \centering
    \includegraphics[width=0.8\textwidth]{plots/1c_2.png}
    %\caption{Number density profile and fit for haloes of approximate mass $10^{13}$ Msun/h.}
    \caption{The same as \ref{fig:ex1c_0}, but for halos of mass $\sim 10^{13}$ Msun/h.}
    \label{fig:ex1c_2}
\end{figure}
\begin{figure}
    \centering
    \includegraphics[width=0.8\textwidth]{plots/1c_3.png}
    %\caption{Number density profile and fit for haloes of approximate mass $10^{14}$ Msun/h.}
    \caption{The same as \ref{fig:ex1c_0}, but for halos of mass $\sim 10^{14}$ Msun/h.}
    \label{fig:ex1c_3}
\end{figure}
\begin{figure}
    \centering
    \includegraphics[width=0.8\textwidth]{plots/1c_4.png}
    %\caption{Number density profile and fit for haloes of approximate mass $10^{15}$ Msun/h.}
    \caption{The same as \ref{fig:ex1c_0}, but for halos of mass $\sim 10^{15}$ Msun/h.}
    \label{fig:ex1c_4}
\end{figure}

The code used for this subquestion is:
\lstinputlisting[firstline=419, lastline=452]{ex1.py}

\subsection*{d}
For this exercise the results of 1b and 1c are compared.
A G-test is used for this.

By calculating the significance Q of G these approaches can be compared.
For the degrees of freedom I used $k=3$, this because we are fitting for 3 parameters (as we compare the model to data at discrete set of x-values, this is not a degree of freedom).

This gave the following values:
\lstinputlisting{output/ex1d.txt}

The values found when minimizing the Poisson log likelihood the highest (and therefore best) Q-values are found.
This is to be expected as the observed data is number of satellite galaxies in at a certain radius.
The actual observations (before taking the mean over a sample of halos) can only take positive integer values, the probability on which is better described by a Poissonian than a Gaussian.
It is notable though, that for halo masses of $\sim 10^{11}$ Msun the Q-values are very low despite figures \ref{fig:ex1b_0} and \ref{fig:ex1c_0} showing a reasonable fit.

The code used for this subquestion is:
\lstinputlisting[firstline=454]{ex1.py}